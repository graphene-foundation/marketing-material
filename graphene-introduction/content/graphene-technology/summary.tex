Graphene is an open source collaborative effort created to advance
cross-industry blockchain technologies. It is a global collaboration,
including leaders in finance, banking, IoT, supply chain, manufacturing
and technology.
Graphene uses P2P technology to operate businesses autonomously
with no central authority. Transactions with instructions to the
business logic are carried out collectively by the entire network.
Graphene is open-source; its design is public and everyone can take
parts and/or contribute. Through many of its unique properties, Graphene allows exciting
uses of Blockchain Technology that could not exist previously including
such as BitShares, Steem, PeerPlays, Muse and others.

Some highlights to conclude this overview:

\paragraph{Protection against fraud}
Any business knows the problem of transaction uniqueness and its audit
trail that are revisited by auditors. Transactions on a Graphene-based
blockchain are irreversible and secure, meaning that the cost of fraud
is no longer pushed onto the shoulders of the business.

\paragraph{Fast and distributed transaction processing that scales}
A Graphene-based blockchain can execute many thousands of transactions
every second in a distributed network, yet reach consensus about your
business logic. In contrast to other blockchain systems, Graphene is
capable of confirming your transactions in single digit seconds.

\paragraph{Multi-signature and accounts}
Graphene blockchains make use of an account-based transaction
processing. This means, participants obtain a named account and can use
it like an email address. Thanks to multi-signature, accounts can be
secured in a way that requires multiple people to approve a transaction
before it becomes valid.

\paragraph{Accounting transparency}
Many organizations are required to produce accounting documents about
their activity. Using a Graphene-based blockchain allows you to offer
the highest level of transparency since you can provide information to
verify balances and transactions through the Blockchain. For example,
non-profit organizations can allow the public to see how much they
received in donations.
\vfill
\begin{center}
\includegraphics[height=2cm]{figures/logo-graphene-blue-invert.png}
\end{center}
\vfill
Do you want to contribute to the Graphene network? Feel free to reach out to the Graphene Foundation and share your ideas or do a proposal. For questions and proposals you can contact us via: \href{mailto:contact@graphene.community}{contact@graphene.community}
