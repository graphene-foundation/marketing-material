The Graphene technology was first used to realize the BitShares Blockchain, which was launched 2015 with its
roots already back in 2013. The BitShares Blockchain utilized the Graphene technology to implement 
high-performance smart contracts with focus on the financial services sector. It has been used for several 
other projects, e.g. Steem, Muse and Peerplays, and recently for EOSIO in the core concepts.

\paragraph{What is Blockchain}
Blockchain, also known as distributed ledger technology, allows a network
of users to keep a record of transactions. In layman's terms, a
blockchain is like an ''Excel sheet'' where each column and row (blocks)
have records (words, numbers or whatever you put into an excel sheet)
recorded. However unlike normal Excel sheets and databases, the blockchain is
distributed,  meaning that it runs across multiple nodes (computers) and
for a new block to be added, all nodes have to "agree" or come to a
consensus. Blockchain technology is the technology of the future and provides a
trust-less platform without the need for a middle man; security through
cryptography and time stamps; and is decentralized which prevents a
single point of failure.

This document serves to introduce the Graphene technology as a framework for realizing a Blockchain, its architecture as well as its governance system using a core native token.
